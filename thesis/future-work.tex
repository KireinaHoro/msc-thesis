\chapter{Future Work}
\begin{chapquote}{Linus Torvalds, \textit{include/math-emu/double.h, Linux-2.6.12-rc2}}
Here's a nickel kid.  Go buy yourself a real computer.
\end{chapquote}

\section{Improving $F_{\text{max}}$} \label{sec:improving-fmax}

An important factor in the current FPsPIN E2E latency as shown in the ICMP ping demo is the handler processing latency.  This is due to the PULP cluster being designed for ASIC and not optimised for maximum $F_{\text{max}}$ on FPGAs.  We plan to integrate higher frequency RISC-V cores designed FPGA, for example VexRiscv\footnote{\url{https://github.com/SpinalHDL/VexRiscv}}, as well as optimise critical paths in the design, to further lower this latency.

\pengcheng{Explore possible timing relaxations such as increasing the SRAM latency}

\section{More Real-world Applications}

We showcased the capabilities of PsPIN handlers through the UDP ping-pong demo.  However, it would be more beneficial to have real-world applications instead of synthetic benchmarks only.  To further explore in this aspect, we are implementing a memcached-style offloaded key-value store application.  We are also working on porting the MPI Datatypes handlers~\cite{Di_Girolamo_2019} to the FPsPIN platform.

\section{Explore Functionalities from Corundum}
\pengcheng{NIC-side DRAM, outbound commands to use checksum offloading, PTP timestamps, etc.}
