\chapter{sPIN Revisited} \label{chap:spin-revisited}
\begin{chapquote}{John Drury Clark, \textit{Ignition!--An Informal History of Liquid Rocket Propellants}}
Their guess turned out to be right, but one is reminded of E. T. Bell's remark that the great vice of the Greeks was not sodomy but extrapolation.
\end{chapquote}

Throughout the process of building the FPsPIN demo, we ...
\pengcheng{intro: missing parts in the sPIN specification for real-world operation.  Communicated back to the sPIN team and some are already adopted into the current sPIN specification}

\section{Handler Initialisation} \label{sec:handler-init}
\pengcheng{How to initialise handler states?  device function that's called once, that is not a handler (no HER meta etc.)}
\pengcheng{Init with host involvement.  Should be defined in sPIN as extended\_init or something}

\section{Streaming Host DMA} \label{sec:streaming-host-dma}
\pengcheng{we currently have one large host DMA region per execution context.  it may be beneficial if we stream host processing by having the app call mmap multiple times to feed a ring buffer in the kernel/hardware to feed to the HER generator so that we have one buffer per message.  How to formalise this in sPIN?}

\section{Packet Matching Rules}

\section{Network-layer Protocol Handling} \label{sec:l3-protocol-handling}
\pengxu{for example incoming ARP probes when using Ethernet/IP}
\pengxu{Sending packet: share neighbour buffer with host?  Implement ARP stack in sPIN?}

\section{Messaging and Reliability Layer: SLMP} \label{sec:slmp}
\pengxu{protocol should be message oriented (instead of TCP).  reliability requirements: SCTP, DCCP, SLMP, pure UDP}
\pengxu{SLMP ACK configuration}

\section{Scheduler Concurrency Control}
\pengxu{Ability to control degree of parallelism \& locality.  Possible implementation: core mask on the scheduler}

\section{Support for Diverse Memory Architectures}
\pengxu{Corundum supports NIC-attached DDR or HBM; should extend the }