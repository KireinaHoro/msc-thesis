\chapter{Conclusion}
\begin{chapquote}{Lewis Carroll, \textit{Alice's Adventures in Wonderland}}
``Begin at the beginning, '' the King said, very gravely, ``and go on till you come to the end: then stop.''
\end{chapquote}

We set out to build FP\acs{spin} as a faster evaluation platform for \ac{spin} handlers that could run the handlers faster than the cycle-accurate simulation from P\acs{spin} and provide a more complete programming model with host applications.  We presented the detailed design and implementation of the hardware (\Cref{chap:hardware}) and software (\Cref{chap:software}) components of the system.  The successful implementation and evaluation of the ping-pong, \ac{mpi} datatypes, and \ac{slmp} file transfer benchmarks (\Cref{chap:eval}) shows that this goal has been sufficiently reached with FP\acs{spin}.

Another important goal of building FP\acs{spin} is to contribute insights and feedback from actually building a \ac{spin} \ac{nic} back to the \ac{spin} specification.  We identified diverse potential improvements, including reliability protocols, telemetry requirements, scheduler improvements, and many more (\Cref{chap:spin-revisited}).  We believe that our feedback would help \ac{spin} become more comprehensive and realistic in achieving its vision for in-network-computing.

Our comprehensive evaluation of FPsPIN through the three benchmarks systematically characterised its performance in latency, throughput, and computation/communication overlap.  We showed that FPsPIN achieved stable latency advantage against the baseline host-only case in the ping-pong demo (\Cref{sec:demos-ping-pong}) and near-perfect overlap in the \ac{mpi} datatypes demo (\Cref{sec:mpi-datatypes-demo}).  We also showed that the real-world throughput of FPsPIN in the \ac{mpi} datatypes benchmark leaves much to be expected compared to the baseline speed test and synthetic \ac{slmp} file transfer demo (\Cref{sec:slmp-demo}), largely due to the low performance of the measly \ac{pulp} cores currently used by FP\acs{spin}.

This shortcoming in throughput provides an outlook of possible improvements to FP\acs{spin} through the integration of a better \ac{hpu} core designed for \ac{fpga}s (\Cref{sec:improving-fmax}).  Even more interestingly, it opens up possible future research in \ac{hpu} architecture on domain-specific acceleration in packet processing (\Cref{sec:hpu-arch}).  We are excited to see future progress in this domain.
